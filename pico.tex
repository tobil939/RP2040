\documentclass[a4paper,12pt,twoside]{article}
\usepackage[ngerman]{babel}
\usepackage[utf8]{inputenc}
\usepackage[T1]{fontenc}
\usepackage{geometry}
\usepackage{fancyhdr}
\usepackage[hidelinks]{hyperref}
\usepackage{listings}
\usepackage{xcolor}
\usepackage{ulem}
\usepackage{graphicx}
\usepackage{amsmath}
\usepackage{hyperref}
\usepackage{caption}
\usepackage{siunitx}

\geometry{
  a4paper, 
  left=3cm,
  right=3cm,
  top=2cm,
  bottom=2cm
}

% Farben für Code
\definecolor{codebg}{RGB}{245,245,245}
\definecolor{keyword}{RGB}{0,0,180}
\definecolor{comment}{RGB}{0,128,0}
\definecolor{string}{RGB}{163,21,21}

\lstset{
  backgroundcolor=\color{codebg},
  basicstyle=\ttfamily\footnotesize,
  keywordstyle=\color{keyword}\bfseries,
  commentstyle=\color{comment},
  stringstyle=\color{string},
  numbers=left,
  numberstyle=\tiny,
  breaklines=true,
  frame=single,
  showstringspaces=false,
  tabsize=4
}

% Titelseite
\title{Raspberry Pi Pico C/C++
       RP2040}
\author{Tobi}
\date{\today}

% Kopf und Fußzeilen
\pagestyle{fancy}
\fancyhf{} % Kopf- und Fußzeilen leeren
\fancyfoot[C]{\thepage} % Seitenzahl zentriert in der Fußzeile
\fancyhead[LE, RO]{\title}
\fancyhead[LO, RE]{\author}


\begin{document}

% Titelseite
\maketitle
\thispagestyle{empty} % Keine Kopf-/Fußzeile auf der Titelseite
\newpage

% Inhaltsverzeichnis
\tableofcontents
\newpage

% Beispielkapitel
\section{Einleitung}
Raspberry Pi Pico rp2040 Chip. Ist eine von der Raspberry Comunity entwickelter Prozessor.
Es gibt auch eine W Variante, dieses hat einen WLan Chip integriert.
Der Prozessor hat zwei Kerne, die separat programmiert werden können.
Hier beziehe ich mich auf die programmierung mit C/C++.
Es ist auch möglich mit Assembler und einer Art von Python den Pico zu programmieren.
\section{Ausführen}
Folgende Schritte werden benötig um den Pico mit einer Software zu flashen.
Auf github.com gibt es eine bash Datei von mir, pico.sh.
Dieses Datei versucht die datei.c zu kompilierern und die datei.uf2 zu erzeugen.
Es gibt eine weiter Datei in meiner github, flash.sh. Dieses bash falsht die Software auf den Pico.
\begin{itemize}
	\item datei.c schreiben
	\item CmakeLists.txt erstellen
	\item \verb|pico_sdk_import.cmake| in das Projektverzeichnis kopieren
	\item build Verzeichnis erstellen
	\item in build wechseln
	\item cmake .. ausführen
	\item make ausführen
	\item Pico in Bootloader starten
	      \begin{itemize}
		      \item Pico spannungsfrei schalten
		      \item Taster auf PCB gedrückt halten
		      \item Pico wieder einschalten
		      \item Pico sollte nun als sdb zu finden sein
	      \end{itemize}
	\item cp datei.uf2 /sdb/..../RPI-RP2
\end{itemize}
\subsection{CMakeLists.txt}
Hier ist eine "Universelle" CMakeLists.txt.
In ihr muss der Dateinamen noch geändert und die Bibliotheken hinzugefügt.
\begin{center}
	\begin{minipage}{1.0\textwidth}
		\begin{lstlisting}[language=C]
cmake_minimum_required(VERSION 3.13)

# Automatisch den SDK-Pfad setzen (falls nicht gesetzt)
if(NOT DEFINED PICO_SDK_PATH)
    set(PICO_SDK_PATH "$ENV{SDK_PATH}")
endif()

if(NOT EXISTS "${PICO_SDK_PATH}")
    message(FATAL_ERROR "PICO_SDK_PATH ist nicht gesetzt oder ungueltig!")
endif()

include(pico_sdk_import.cmake)

# Projektname setzen
project(multi C CXX ASM)

set(CMAKE_C_STANDARD 11)
set(CMAKE_CXX_STANDARD 17)

pico_sdk_init()

# Alle C- und C++-Dateien automatisch erkennen
file(GLOB SRC_FILES *.c *.cpp)

add_executable(${PROJECT_NAME} ${SRC_FILES})

# Bibliotheken automatisch einbinden
target_link_libraries(${PROJECT_NAME}
    pico_multicore
    pico_stdlib
    hardware_pio
)

# Falls USB genutzt wird
if(EXISTS "${CMAKE_CURRENT_LIST_DIR}/usb.c" OR EXISTS "${CMAKE_CURRENT_LIST_DIR}/usb.cpp")
    target_link_libraries(${PROJECT_NAME} tinyusb_device tinyusb_host)
endif()

# Falls Floating Point-Unterstuetzung benoetigt wird
target_link_libraries(${PROJECT_NAME} pico_float pico_double)

# Falls C++ Standardbibliothek genutzt wird
if(CMAKE_CXX_COMPILER)
    target_link_libraries(${PROJECT_NAME} pico_standard_link)
endif()

# USB- oder UART-Standardausgabe aktivieren
pico_enable_stdio_usb(${PROJECT_NAME} 1)
pico_enable_stdio_uart(${PROJECT_NAME} 0)

# Extra-Ausgaben fuer Flash
pico_add_extra_outputs(${PROJECT_NAME})

# Generiere PIO-Header fuer gefundene .pio-Dateien
file(GLOB PIO_FILES *.pio)
foreach(pio_file ${PIO_FILES})
    pico_generate_pio_header(${PROJECT_NAME} ${pio_file})
endforeach()
	  \end{lstlisting}
	\end{minipage}
\end{center}
\section{Programmierung}
\subsection{Bibliotheken}
hier ein paar Links in denen man die meisten Bibliotheken anschauen lann.
\paragraph{Pfade github.com}
%\label{url:bib1}
\small \protect\href{https://github.com/raspberrypi/pico-sdk/tree/master/src/rp2_common}{Quelle: github.com}
\paragraph{Pfad lorenz-ruprecht.at}
%\label{url:bib2}
\small \protect\href{https://lorenz-ruprecht.at/docu/pico-sdk/1.4.0/html/group__hardware.html}{Quelle: lorenz-ruprecht.at}
\paragraph{Home-Verzeichnis}
%\label{url:bib3}
\small \texttt{$~/pico-sdk/src/rp2_common/$}
\section{Debugger}
\subsection{Benötigte Hardware}
Für das Debuggin wird eine Raspberry Pi Debug Probe benötigt.
\subsection{Benötigte Programme}
Es wird spezeill für eine Linux-System mit NeoVim beschrieben.
\begin{itemize}
	\item GDB
	\item arm-none-eabi-gcc
	\item amr-none-eabi-gdb
	\item openocd
	\item openocd-picoprobe
\end{itemize}
\subsection{OpenOCD}
\verb|openocd -f interface/picoprobe.cfg -f target/rp2040.cfg|
\verb|interface/picoprobe.cfg| Konfiguriert den Picoprobe-Adapter
\verb|target/rp2040.cfg| Definiert den RP2040 Mikrocontroller
\verb|Info : Listening on port 3333 for gdb connections| das sollte dann die
\subsection{GDB im Terminal} \label{subsec:terminal}
\begin{tabbing}
	\hspace{5mm} \= \hspace{50mm} \= \kill
	\>  \hspace{2mm}  starten des OpenOCD, Verbindung aufbauen \\
	\> \verb|openocd -f interface/picoprobe.cfg -r target/rp2040.cfg| \\
	\> \hspace{2mm} starten des GDB im Terminal \\
	\> \verb|arm-none-eabi-gdb build/mein_projekt.elf| \\
	\>  \hspace{2mm}verbinden von GDB mit OpenOCD \\
	\> \verb|target remote localhost:3333| \\
	\> \verb|load| \> laden des Programms auf den Pico \\
	\> \verb|run| \> startet das Programm vom Anfang an \\
	\> \verb|continue| \> führt bis zum nächsten Breakpoint weiter \\
	\> \verb|step| \> führt die nächste Programmzeile aus \\
	\> \verb|next| \> führt die nächste Programmzeile aus, keine Funktionen \\
	\> \verb|finish| \> führt Funktion bis zum Ende aus \\
	\> \verb|kill| \> beendet das aktuelle Programm \\
	\> \verb|quit| \> verlässt GDB \\
	\> \verb|break bunga| \> setzt Breakpoint bei der Funktion bunga \\
	\> \verb|break datei.c:52| \> setzt Breakpoint bei der Zeile 52 \\
	\> \verb|info breakpoints| \> listet alle Breakpoints auf \\
	\> \verb|delete 4| \> löscht Breakpoint Nr. 4 \\
	\> \verb|disable 2| \> deaktiviert Breakpoint Nr. 2 \\
	\> \verb|enable 2| \> aktiviert Breakpoint Nr. 2 \\
	\> \verb|print meine_variable| \> gibt Wert von \verb|meiner_variable| aus \\
	\> \verb|print *pointer| \> gibt des Wert des Pointers aus \\
	\> \verb|display meine_variable| \> zeigt den Wert der Variable bei jedem Stop \\
	\> \verb|info locals| \> listet alle Variablen auf \\
	\> \verb|info registers r0| \> Zeigt die Werte des Registers r0 \\
	\> \verb|backtrace| \> zeigt den Call Stack an \\
	\> \verb|frame 1| \> wechselt vom Frame 1 auf den backtrace \\
	\> \verb|list bunga| \> zeigt Code um die \verb|bunga-Funktion| \\
	\> \verb|set meine_variable = 32| \> setzt \verb|meine_variable| auf 32 \\
	\> \verb|watch meine_variable| \> hält an wenn \verb|meine_variable| verändert wird \\
	\> \verb|advance datei.c:50| \> führt das Programm bis Zeile 50 aus \\
\end{tabbing}
\subsection{GDB im NeoVim}
\begin{tabbing}
	\hspace{5mm} \= \hspace{50mm} \= \kill
	\> \verb|:vs| \> öffnet ein neues Fenster \\
	\> \verb|:term| \> öffnet das Terminal in NeoVim \\
	\> \hspace{2mm}In diesem Terminal können dann alle Befehle aus \ref{subsec:terminal} verwendet werden. \\
	\> \verb|:DapContinue| \> Startet oder führt das Programm weiter aus \\
	\> \verb|:DapStepOver| \> führt nächste Zeile aus, ohne Funktionen \\
	\> \verb|:DapStepInto| \> sprint in Funktion \\
	\> \verb|:DapStepOut| \> verlässt die Funktion \\
	\> \verb|:DapRestart| \> startet das Programm neu \\
	\> \verb|:DapTerminate| \> beendet das Debugging \\
	\> \verb|:DapToggleBreakpoint| \> setzt oder entfernt einen Breakpoint \\
	\> \hspace{2mm}hält an wenn counter 5 ist \\
	\> \verb|:DapSetBreakpoint counter == 5| \\
	\> \verb|:DapClearBreakpoints| \> entfernt alle Breakpoints \\
	\> \verb|:DapHover| \> zeigt den Wert der Variable unter dem Cursor \\
	\> \verb|:DapScopes| \> öffnet ein Fenster mit den Variablen \\
	\> \verb|:DapFrames| \> zeigt den Call Stack \\
	\> \hspace{2mm}gibt alle Variablen im aktuellen Scope aus \\
	\> \verb|:lua print(vim.inspect(dap.session().variables))| \\
	\> \verb|:DapUiToggle| \> schalter UI ein oder aus \\
\end{tabbing}
\section{I/O}
\subsection{Digital}
\subsubsection{Eigenschaften}
\begin{center}
	\begin{tabular}{|c|c|c|}
		\hline
		     & \textbf{Eingang} & \textbf{Ausgang} \\
		\hline
		\hline
		high & $> 2,0V$         & $\approx 3,3V$   \\
		\hline
		low  & $< 0,8V$         & $\approx 0,0V$   \\
		\hline
	\end{tabular}
\end{center}
\subsubsection{Layout}
\begin{itemize}
	\item \textbf{GP0 - GP22}: 23 digitale GPIOs
	\item \textbf{GP25}: Onboard-LED
	\item \textbf{GP26 - GP28}: 3 analoge Pins (können auch digital genutzt werden)
\end{itemize}
\subsubsection{Bibliotheken}
\verb|"pico/stdlib.h"| \\
\verb|"hardware/gpio.h"| \\
\subsubsection{Initialisierung und Konfiguration}
\begin{tabbing}
	\hspace{5mm} \= \hspace{80mm} \= \kill
	\> \verb|const uint BUNGA = 2 | \> definiert BUNGA als Pin 2 \\
	\> \verb|gpio_init(BUNGA);| \> definiert Bunga \\
	\> \verb|gpio_set_dir(BUNGA, GPIO_OUT);| \> definiert Bunga als Ausgang \\
	\> \verb|gpio_set_dir(BUNGA, GPIO_IN);| \> definiert Bunga als Eingang \\
	\> \verb|gpio_set_pulls(BUNGA, true, false);| \> aktiviert pull up \\
	\> \verb|gpio_set_pulls(BUNGA, false, true);| \> aktiviert pull down \\
	\> \verb|gpio_set_pulls(BUNGA, false, false);| \> deaktiviert pull up/down \\
	\> \verb|gpio_pull_up(BUNGA);| \> aktiviert pull up \\
	\> \verb|gpio_pull_down(BUNGA);| \> aktiviert pull down \\
	\> \verb|gpio_disable_pulls(BUNGA);| \> deaktiviert pull up/down \\
	\> \verb|gpio_put(BUNGA, true);| \> schaltet Bunga auf high \\
	\> \verb|gpio_put(BUNGA, false);| \> schaltet Bunga auf low \\
	\> \verb|gpio_get(BUNGA);| \> liest Zustand von Bunga \\
\end{tabbing}
\subsubsection{Beispiel}
\begin{center}
	\begin{minipage}{1.0\textwidth}
		\begin{lstlisting}[language=C]
#include "pico/stdlib.h"
#include "hardware/gpio.h"

#define lampe1 18  // externe LED
#define taster2 21  // Taster

int main(){
  gpio_init(lampe1);
  gpio_init(taster2);

  gpio_set_dir(lampe1, GPIO_OUT);
  gpio_set_dir(taster2, GPIO_IN);

  gpio_pull_up(taster2);
  bool taster2_state = gpio_get(taster2);
  gpio_put(lampe1, 1);   // lampe1 an
}
	  \end{lstlisting}
	\end{minipage}
\end{center}
\subsection{Analog}
\subsubsection{Eingang}
\paragraph{Eigenschaften}
\begin{center}
	\begin{tabular}{|c|c|}
		\hline
		Eingang            &                     \\
		\hline
		\hline
		Auflösung          & 12bit $(0-4095)$    \\
		\hline
		Eingangsspannung   & $0V bis 3,3V$       \\
		\hline
		Referenzspannung   & $3,3V$              \\
		\hline
		Eingangswiderstand & $\approx 1M \Omega$ \\
		\hline
		Sampling Rate      & max. $500kS/s$      \\
		\hline
		1 Kanal            & $500kS/s$           \\
		\hline
		2 Kanal            & $250kS/s$           \\
		\hline
		3 Kanal            & $166kS/s$           \\
		\hline
	\end{tabular}
\end{center}
\subsubsection{Layout}
\begin{center}
	\begin{tabular}{|c|c|c|}
		\hline
		Name   & Pin    & Bezeichnung            \\
		\hline
		\hline
		GPIO26 & Pin 31 & ADC 0 Eingang          \\
		\hline
		GPIO27 & Pin 32 & ADC 1 Eingang          \\
		\hline
		GPIO28 & Pin 34 & ADC 2 Eingang          \\
		\hline
		       &        & ADC 3 nicht verfügbar  \\
		\hline
		       &        & ADC 4 Temperaturfühler \\
		\hline
	\end{tabular}
\end{center}
\paragraph{Berechnung der Sample Rate}
Jede Messung benötigt mindestes 96 Zyklen pro Sample \\
\begin{align}
	\text{Sample Rate} & = \frac{\text{ADC-Takt}}{\text{Zyklen pro Sample}} \\
	                   & = \frac{48\,\text{MHz}}{96}                        \\
	                   & = 500\,\text{kS/s}
\end{align}
\begin{align}
	\text{Sample Rate} & = \frac{\text{ADC-Takt}}{\text{clkdiv} \times \text{Zyklen pro Sample}} \\
	                   & = \frac{48\,\text{MHz}}{100 \ times 96}                                 \\
	                   & = 5\,\text{kS/s}
\end{align}
\subsubsection{Bibliotheken}
\verb|"pico/stdlib.h"| \\
\verb|"hardware/adc.h"| \\
\subsubsection{Initialisierung und Konfiguration}
\begin{tabbing}
	\hspace{5mm} \= \hspace{80mm} \= \kill
	\> \verb|void asd_init(void)| \> initialisiert ADC Hardware \\
	\> \verb|void adc_gpio_init(unit gpio)| \> initialisiert einen GPIO Pin \\
	\> \verb|void adc_select_input(uint gpio)| \> wählt GPIO Pin als Messsignal \\
	\> \verb|uint adc_get_selected_input(void)| \> aktuellen Wert einlesesn \\
	\> \verb|void adc_set_round_robin(uint input_mask)| \\
	\> \hspace{2mm}liest alle ADC ein \\
	\> \verb|input_mask| \> 0b00000101 für ADC0 und ADC2 \\
	\> \verb|void adc_set_temp_sensor_enabled(bool)| \> interner Temperaturfüheler aktivieren true/false \\
	\> \verb|uint16_t adc_read(void)| \> liest ADC ein \\
	\> \verb|void adc_run(bool)| \> "freier Lauf" der Messung true/false \\
	\> \verb|void adc_set_clkdiv(float clkdiv)| \> setzt die Sampe Rate \\
	\> \verb|void adc_fifo_setup(en, dreq_en, dreq_thresh, err_in_fifo, byte_shift)| \\
	\> \verb|bool adc_fifo_is_empty(void)| \> rückgabe von true wenn fifo leer ist \\
	\> \verb|uint8_t adc_fifo_get_level(void)| \> rückgabe der Anzahl von Einträgen \\
	\> \verb|uint16_t adc_fifo_get_blocking(void)| \> wartet bis Daten in fifo \\
	\> \verb|void adc_fifo_drain(void)| \> leert fifo wenn der letzte Zugriff beendet \\
	\> \verb|void adc_irq_set_enabled(bool)| \> true/false von Interruprs \\
\end{tabbing}
\subsubsection{FIFO}
FIFO steht für "First In, First Out" und ist eine Art Pufferspeicher für ADC Werte.
Es können 4 x 12bit Werte gespeichert werden. Mit DMA "Direct Memory Access" können Messwerte,
auch automatisch in einen Speicher übertragen werden ohne größe CPU Auslastung.
\paragraph{adc\_fifo\_setup}
\begin{center}
	\begin{minipage}{1.0\textwidth}
		\begin{lstlisting}[language=C]
adc_fifo_setup(
  en,           // bool FIFO aktivieren 
  dreq_en,      // bool DMA anfragen wenn Daten vorhanden
  dreq_thresh,   // uint16_t DMA oder IRQ anfrage wenn erreicht
  err_in_fifo,  // bool bit 15 ist das error flag fuer jedes Sample
  byte_shift    // bool Fifo von 12bit auf 8bit fuer DMA
);
	  \end{lstlisting}
	\end{minipage}
\end{center}
\subsubsection{Beispiel}
\begin{center}
	\begin{minipage}{1.0\textwidth}
		\begin{lstlisting}[language=C]
#include <stdio.h>
#include "pico/stdlib.h"
#include "hardware/adc.h"

int main() {
    stdio_init_all();
    
    adc_init();  // ADC aktivieren
    adc_gpio_init(26);  // GPIO26 (ADC0) aktivieren
    adc_select_input(0);  // ADC-Kanal 0 waehlen (entspricht GPIO26)

    while (1) {
        uint16_t result = adc_read();  // 12-Bit-Wert (0 bis 4095)
        float voltage = result * 3.3f / (1 << 12);  // Spannung berechnen
        printf("ADC-Wert: %d, Spannung: %.2fV\n", result, voltage);
        sleep_ms(1000);
    }
}
	  \end{lstlisting}
	\end{minipage}
\end{center}
\subsubsection{Temperaturfühler}
\paragraph{Befehle}
\begin{tabbing}
	\hspace{5mm} \= \hspace{80mm} \= \kill
	\> \verb|adc_init();| \> ADC wird aktiviert \\
	\> \verb|adc_set_temp_sensor_enable(true);| \> aktiviert den Termperatursensor \\
	\> \verb|adc_select_input(4);| \> wählt ADC4 \\
	\> \verb|adc_read();| \> liest den ADC Wert (12bit 0 bis 4095) \\
\end{tabbing}
\paragraph{Berechnung}
\begin{align}
	\text{Spannung} & = \frac{\text{ADC-Wert}\times 3,3\text{V}}{\text{12bit}} \\
	                & = \frac{\text{ADC}\times 3,3\text{V}}{4096}              \\
\end{align}
\begin{align}
	\text{Temperatur} & = {27} - \frac{\text{Spannung}- \text{Ausgangsspannung bei 27*C}}{ \text{Temperaturdrift} } \\
	                  & = {27} - \frac{\text{V}- 0,706}{0,001721}                                                   \\
\end{align}
\paragraph{Beispiel}
\begin{center}
	\begin{minipage}{1.0\textwidth}
		\begin{lstlisting}[language=C]
#include <stdio.h>
#include "pico/stdlib.h"
#include "hardware/adc.h"

int main() {
		stdio_init_all();  // UART fuer die serielle Ausgabe initialisieren

		adc_init();  // ADC aktivieren
		adc_set_temp_sensor_enabled(true);  // Internen Temperatursensor aktivieren
		adc_select_input(4);  // ADC4 ist der interne Temperatursensor

		while (1) {
				uint16_t raw = adc_read();  // 12-Bit-ADC-Wert auslesen (0 bis 4095)
				float voltage = raw * 3.3f / (1 << 12);  // In Spannung umwandeln
				float temperature = 27.0f - (voltage - 0.706f) / 0.001721f;  // Temperatur berechnen

				printf("Temperatur: %.2f C\n", temperature);  // Ausgabe ueber UART
				sleep_ms(1000);  // 1 Sekunde warten
			}
	}
\end{lstlisting}
	\end{minipage}
\end{center}
\subsubsection{Ausgang}
\paragraph{Eigenschaften}
Es gibt keine direkten Analogausgang, man muss über ein PWM Signal eine Spannung erzeugen.\\
Am besten noch einen Kondensator dazwischen um den Mittelwert zu erhalten. \\
Es gibt auch ein paar ICs die man über I2C oder SPI ansteuern kann die die Funktiin eines DAC haben. \\
\begin{center}
	\begin{tabular}{|c|c|}
		\hline
		Ausgang (PWM)      &                               \\
		\hline
		\hline
		Auflösung          & 16bit $(0-65535)$             \\
		\hline
		Ausgangsspannung   & $0,0V bis 3,3V$               \\
		\hline
		Ausgangswiderstand & $30\Omega$ bis $50\Omega$     \\
		\hline
		Ausgangsstrom      & $12mA$ pro Kanal, max. $50mA$ \\
		\hline
	\end{tabular}
\end{center}
\paragraph{Berechnung Tiefpassfilter}
Diese Berechnung dient dazu den Tiefpassfilter für das die Umwandlung des PWM zum DAC Signal. \\
Am besten einen Widerstand in der Größenordnung von $460k\Omega$ bis $820k\Omega$ wählen. \\
\paragraph{Formel}
\begin{align}
	C & = \frac{1}{2 \pi R f}
\end{align}
\begin{itemize}
	\item \{ C \}: Kapazitaet in \si{\farad}
	\item \{ R \}: Widerstand in \si{\ohm}
	\item \{ f \}: PWM-Frequenz in \si{\hertz}
\end{itemize}
\begin{center}
	\begin{figure}[h]
		\centering
		\includegraphics[width=1\textwidth]{/home/tobil/Documents/prog/pico/pic/pic4.png}
		\caption{Tiefpassfilter}
		\small \href{https://www.heise.de/hintergrund/Gaengige-Verfahren-zur-Digital-Analog-Wandlung-6363918.html?seite=2}{Bildquelle: heis.de}
		%\label{fig:pic4}
	\end{figure}
\end{center}
\subsubsection{Auflösung PWM}
\begin{align}
	\text{PWM-Frequenz} & = \frac{\text{Systemtakt}}{\text{Taktteiler} \times (\text{Zählgrenze} + 1)} \\
	                    & = \frac{125\,000\,000}{4 \times (255 + 1)}                                   \\
	                    & = 122{,}07\,\text{kHz}
\end{align}
\begin{center}
	\begin{tabular}{|c|c|c|}
		\hline
		Auflösung & Zählergrenze & PWM (bei 125MHz, \textbf{Teiler 1}) \\
		\hline
		\hline
		8 bit     & 255+1        & 488,3kHz                            \\
		\hline
		10 bit    & 1023+1       & 122,1kHz                            \\
		\hline
		12 bit    & 4095+1       & 30,5kHz                             \\
		\hline
		16 bit    & 65535+1      & 1,9kHz                              \\
		\hline
	\end{tabular}
	\\[10mm]
	\begin{tabular}{|c|c|c|}
		\hline
		Auflösung & Zählergrenze & PWM (bei 125MHz, \textbf{Teiler 4}) \\
		\hline
		\hline
		8 bit     & 255+1        & 122,1kHz                            \\
		\hline
		10 bit    & 1023+1       & 30,5kHz                             \\
		\hline
		12 bit    & 4095+1       & 7,6kHz                              \\
		\hline
		16 bit    & 65535+1      & 0,48kHz                             \\
		\hline
	\end{tabular}
	\\[10mm]
	\begin{tabular}{|c|c|c|}
		\hline
		Auflösung & Zählergrenze & PWM (bei 125MHz, \textbf{Teiler 100}) \\
		\hline
		\hline
		8 bit     & 255+1        & 4,9kHz                                \\
		\hline
		10 bit    & 1023+1       & 1,2kHz                                \\
		\hline
		12 bit    & 4095+1       & 0,31kHz                               \\
		\hline
		16 bit    & 65535+1      & 0,02kHz                               \\
		\hline
	\end{tabular}
	\\[10mm]
	\begin{tabular}{|c|c|c|}
		\hline
		Auflösung & Zählergrenze & PWM (bei 125MHz, \textbf{Teiler 255}) \\
		\hline
		\hline
		8 bit     & 255+1        & 1914,8Hz                              \\
		\hline
		10 bit    & 1023+1       & 478,7Hz                               \\
		\hline
		12 bit    & 4095+1       & 119,6Hz                               \\
		\hline
		16 bit    & 65535+1      & 7,5Hz                                 \\
		\hline
	\end{tabular}
\end{center}
\paragraph{Layout}
siehe \ref{subsec:terminal}
\subsubsection{Bibliotheken}
\verb|"pico/stdlib.h"| \\
\verb|"hardware/pwn.h"| \\
\small \href{https://github.com/raspberrypi/pico-sdk/blob/master/src/rp2_common/hardware_pwm/include/hardware/pwm.h}{github.com pwm.h}
\subsubsection{Initialisierung und Konfiguration}
\begin{tabbing}
	\hspace{5mm} \= \hspace{80mm} \= \kill
	\> \verb|uint pwm_gpio_to_slice_num(uint gpio)| \> gibt den Slice des GPIO zurück \\
	\> \verb|uint pwm_gpio_to_channel(uint gpio)| \> gibt den Kanal des GPIO zurück \\
	\> \verb|void pwm_config_set_phase_correct(pwm_config *c, float div)| \\
	\> \hspace{2mm} set phase correction in a PWM configuration \\
	\> \verb|void pwm_config_set_clkdiv(pwm_config *c, float div| \\
	\> \hspace{2mm} set PWM clock divider in a PWM configuration \\
	\> \verb|void pwm_config_set_clkdiv_int_fac(pwm_config *c, uint8_t integer, uint8_t fract)| \\
	\>  \hspace{2mm} set PWM clock divider in a PWM configuration usidn a 8:4 fractional value \\
	\> \verb|void pwm_config_set_clkdiv_int(pwm_config *c, uint div)| \\
	\>  \hspace{2mm} set PWM clock divider in a PWM configuration \\
	\> \verb|void pwm_config_set_clkdiv_mode(pwm_config *c, enum pwm_clkdiv_mode mode)| \\
	\>  \hspace{2mm} set PWM counting mode in a PWM configuration \\
	\> \verb|void pwm_condig_set_output_polarity(pwm_config *c, bool a, bool b)| \\
	\>  \hspace{2mm} set output polarity in a PWM configuration \\
	\> \verb|void pwm_config_set_wrap(pwm_config *c, uint16_t wrap)| \\
	\>  \hspace{2mm} set PWM counter wrap value in a PWM configuration \\
	\> \verb|void pwm_init(uint slice_num, pwm_config *c, bool start)| \\
	\>  \hspace{2mm} initialise a PWM with settings from a configuration object \\
	\> \verb|pwm_config pwm_get_default_config(void)| \> standart Werte für eine PWM Konfiguration \\
	\> \verb|void pwm_set_wrap(uint slice_num, uint16_t wrap)| \> setzt den aktuellen PWM Zähler \\
	\> \verb|void pwm_set_chan_level(uint slice_num, uint chan, uint16_t level)| \\
	\>  \hspace{2mm} set the current PWM counter compare value for one channel A or B\\
	\> \verb|void pwm_set_both_levels(uint slice_num, uint16_t level_a, uint16_t level_b)| \\
	\>  \hspace{2mm} set PWM counter compare values \\
	\> \verb|void pwm_set_gpio_level(uint gpio, uint16_t level)| \\
	\>  \hspace{2mm} helper function to set the PWM level for the slice and channel associated with a GPIO \\
	\> \verb|uint16_t pwm_get_counter(uint slice_num)| \> aktuellen Wert des PWM Zähler \\
	\> \verb|void pwm_set_counter(uint slice_num)| \> setzte den PWM Zähler \\
	\> \verb|void pwm_advance_counter(uint slice_num)| \> advance the phase of a running counter by 1 \\
	\> \verb|void pwm_retard_count(uint slice_num)| \> retard the phase of a running counter by 1 \\
	\> \verb|void pwm_set_clkdiv_int_fra(uint slice_num, uint8_t integer, uint8_t frac)| \\
	\>  \hspace{2mm} set PWM clock divider using an 8:4 fraction value \\
	\> \verb|void pwm_set_clkdiv(uint slice_num, float divider)| \> setzt Teiler \\
	\> \verb|void pwm_set_output_polarity(uint slice_num, bool a, bool b)| \\
	\>  \hspace{2mm} setzt die Polarität des Ausgangs \\
	\> \verb|void pwm_set_clkdiv_mode(uint slice_num, enum pwm_clkdiv_mode mode)| \\
	\>  \hspace{2mm} setzt den PWM Teiler-Modus \\
	\> \verb|void pwm_set_phase_correct(uint slice_num, bool phase_correct)| \\
	\>  \hspace{2mm} setzt den Phasen Korrekturwert \\
	\> \verb|void pwm_set_enabled(uint slice_num, bool eanabled)| \\
	\>  \hspace{2mm} aktiviert/deaktiviert PWM Ausgang \\
	\> \verb|void pwm_set_mask_enabled(uint32_t mask)| \\
	\>  \hspace{2mm} aktiviert/deaktiviert mehrere PWM Ausgänge \\
	\> \verb|void pwm_set_irq_enabled(uint slice_num, bool enabled)| \\
	\>  \hspace{2mm} aktiviert/deaktiviert den Interrupt eines PWM Slice \\
	\> \verb|void pwm_set_irq_mask_enabled(uint32_t slice_mask, bool enabled)| \\
	\>  \hspace{2mm} aktiviert/deaktiviert mehrere Interrupts von PWM Slices \\
	\> \verb|void pwm_clear_irq(uint slice_num)| \> setzt einen Interrupt zurück \\
	\> \verb|uint32_t pwm_get_irq_status_mask(void)| \> liest den Interrupt aus \\
	\> \verb|void pwm_force_irq(uint slice_num)| \> setzt eine Interrupt \\
	\> \verb|uint pwm_get_dreq(uint slice_num)| \\
	\>  \hspace{2mm} Return the DREQ to use for pacing transfers to a particular PWM slice \\
\end{tabbing}
\subsubsection{Beispiel}
		\paragraph{pwm\_config}
\begin{center}
	\begin{minipage}{1.0\textwidth}
		\begin{lstlisting}[language=C]
    \\ Standard PWM Konfiguration laden
    pwm_config bunga = pwm_get_default_config();

    \\ Teiler definieren 4.0f fuer float
    pwm_config_set_clkdiv(&bunga, 4.0f);

    \\ max Zaehler definieren
    pwm_config_set_wrap(&bunga, 1023);

    \\ config laden
    pwm_init(slice_num, &bunga, true);
    \end{lstlisting}
		%\label{prog:pwm_config}
	\end{minipage}
\end{center}
	\paragraph{kleinPWM}
\begin{center}
	\begin{minipage}{1.0\textwidth}
		\begin{lstlisting}[language=C]
#include "pico/stdlib.h"
#include "hardware/pwm.h"

int main() {
    gpio_set_function(16, GPIO_FUNC_PWM);  // GPIO16 als PWM-Ausgang setzen
    uint slice_num = pwm_gpio_to_slice_num(16);  // Slice-Nummer fuer GPIO16 holen

    pwm_set_wrap(slice_num, 255);  // Wrap-Wert setzen (Zaehlgrenze = 255)
    pwm_set_clkdiv(slice_num, 4.0);  // Taktteiler = 4.0 (125 MHz / 4)
    pwm_set_chan_level(slice_num, PWM_CHAN_A, 128);  // 50% Duty Cycle (128 / 255)
    
    pwm_set_enabled(slice_num, true);  // PWM aktivieren

    while (1) {
        sleep_ms(1000);
    }
}

    \end{lstlisting}
		%\label{prog:kleinPWM}
	\end{minipage}
\end{center}
		\paragraph{großPWM}
\begin{center}
	\begin{minipage}{1.0\textwidth}
		\begin{lstlisting}[language=C]
    #include "pico/stdlib.h"
#include "hardware/pwm.h"

int main() {
    // Zwei GPIO-Pins fuer PWM (z. B. GPIO 14 und GPIO 15)
    uint pin1 = 14; // Slice 7, Kanal A
    uint pin2 = 15; // Slice 7, Kanal B

    // Zuordnung zu PWM-Slices
    uint slice_num1 = pwm_gpio_to_slice_num(pin1);
    uint slice_num2 = pwm_gpio_to_slice_num(pin2); // Wird dasselbe Slice sein, da 14 und 15 zu Slice 7 gehoeren

    // Zwei benannte pwm_config-Instanzen erstellen
    pwm_config config_slow = pwm_get_default_config(); // Fuer langsames Blinken
    pwm_config config_fast = pwm_get_default_config(); // Fuer schnelles Blinken

    // Konfiguration fuer langsames Blinken (niedrige Frequenz)
    pwm_config_set_clkdiv(&config_slow, 100.0f);  // Systemclock / 100 = 1.25 MHz
    pwm_config_set_wrap(&config_slow, 1249);      // Frequenz = 1.25 MHz / 1250 = 1 kHz
    pwm_config_set_phase_correct(&config_slow, false);

    // Konfiguration fuer schnelles Blinken (hoehere Frequenz)
    pwm_config_set_clkdiv(&config_fast, 4.0f);    // Systemclock / 4 = 31.25 MHz
    pwm_config_set_wrap(&config_fast, 624);       // Frequenz = 31.25 MHz / 625 = 50 kHz
    pwm_config_set_phase_correct(&config_fast, false);

    // GPIO-Pins als PWM konfigurieren
    gpio_set_function(pin1, GPIO_FUNC_PWM);
    gpio_set_function(pin2, GPIO_FUNC_PWM);

    // Initialisiere die PWM-Slices mit den benannten Konfigurationen
    pwm_init(slice_num1, &config_slow, false); // Slice 7 mit langsamer Konfig
    pwm_init(slice_num2, &config_fast, false); // Dasselbe Slice, wird ueberschrieben (siehe Hinweis unten)

    // Setze unterschiedliche Duty Cycles
    pwm_set_gpio_level(pin1, 625);  // 50% Duty Cycle fuer pin1
    pwm_set_gpio_level(pin2, 312);  // 50% Duty Cycle fuer pin2 (angepasst an Wrap-Wert von config_fast)

    // Starte beide PWM-Slices
    pwm_set_enabled(slice_num1, true);

    while (true) {
        // Endlosschleife, PWM laeuft weiter
        sleep_ms(1000);
    }

    return 0;
}
    \end{lstlisting}
		%\label{prog:großPWM}
	\end{minipage}
\end{center}
\subsection{Layout}
\begin{center}
	\begin{figure}[h]
		\centering
		\includegraphics[width=1\textwidth]{/home/tobil/Documents/prog/pico/pic/pic3.png}
		\caption{Raspberry Pi Pico Layout}
		\small \href{https://www.raspberrypi.com/documentation/microcontrollers/pico-series.html}{Bildquelle: raspberry.com}
		%\label{fig:pic3}
	\end{figure}
\end{center}
\newpage
\begin{center}
	\resizebox{\textwidth}{!}{%
		\begin{tabular}{|c|*{9}{c|}}
			\hline
			\textbf{GPIO} & \textbf{F1} & \textbf{F2} & \textbf{F3} & \textbf{F4} & \textbf{F5} & \textbf{F6} & \textbf{F7} & \textbf{F8}  & \textbf{F9}   \\
			              &             &             &             &             &             &             &             &              &               \\ \hline
			\textbf{0}    & SPI0 RX     & UART0 TX    & I2C0 SDA    & PWM0 A      & SIO         & PIO0        & PIO1        &              & USB OVCUR DET \\
			              &             &             &             &             &             &             &             &              &               \\ \hline
			\textbf{1}    & SPI0 CSn    & UART0 RX    & I2C0 SCL    & PWM0 B      & SIO         & PIO0        & PIO1        &              & USB VBUS DET  \\
			              &             &             &             &             &             &             &             &              &               \\ \hline
			\textbf{2}    & SPI0 SCK    & UART0 CTS   & I2C1 SDA    & PWM1 A      & SIO         & PIO0        & PIO1        &              & USB VBUS EN   \\
			              &             &             &             &             &             &             &             &              &               \\ \hline
			\textbf{3}    & SPI0 TX     & UART0 RTS   & I2C1 SCL    & PWM1 B      & SIO         & PIO0        & PIO1        &              & USB OVCUR DET \\
			              &             &             &             &             &             &             &             &              &               \\ \hline
			\textbf{4}    & SPI0 RX     & UART1 TX    & I2C0 SDA    & PWM2 A      & SIO         & PIO0        & PIO1        &              & USB VBUS DET  \\
			              &             &             &             &             &             &             &             &              &               \\ \hline
			\textbf{5}    & SPI0 CSn    & UART1 RX    & I2C0 SCL    & PWM2 B      & SIO         & PIO0        & PIO1        &              & USB VBUS EN   \\
			              &             &             &             &             &             &             &             &              &               \\ \hline
			\textbf{6}    & SPI0 SCK    & UART1 CTS   & I2C1 SDA    & PWM3 A      & SIO         & PIO0        & PIO1        &              & USB OVCUR DET \\
			              &             &             &             &             &             &             &             &              &               \\ \hline
			\textbf{7}    & SPI0 TX     & UART1 RTS   & I2C1 SCL    & PWM3 B      & SIO         & PIO0        & PIO1        &              & USB VBUS DET  \\
			              &             &             &             &             &             &             &             &              &               \\ \hline
			\textbf{8}    & SPI1 RX     & UART1 TX    & I2C0 SDA    & PWM4 A      & SIO         & PIO0        & PIO1        &              & USB VBUS EN   \\
			              &             &             &             &             &             &             &             &              &               \\ \hline
			\textbf{9}    & SPI1 CSn    & UART1 RX    & I2C0 SCL    & PWM4 B      & SIO         & PIO0        & PIO1        &              & USB OVCUR DET \\
			              &             &             &             &             &             &             &             &              &               \\ \hline
			\textbf{10}   & SPI1 SCK    & UART1 CTS   & I2C1 SDA    & PWM5 A      & SIO         & PIO0        & PIO1        &              & USB VBUS DET  \\
			              &             &             &             &             &             &             &             &              &               \\ \hline
			\textbf{11}   & SPI1 TX     & UART1 RTS   & I2C1 SCL    & PWM5 B      & SIO         & PIO0        & PIO1        &              & USB VBUS EN   \\
			              &             &             &             &             &             &             &             &              &               \\ \hline
			\textbf{12}   & SPI1 RX     & UART0 TX    & I2C0 SDA    & PWM6 A      & SIO         & PIO0        & PIO1        &              & USB OVCUR DET \\
			              &             &             &             &             &             &             &             &              &               \\ \hline
			\textbf{13}   & SPI1 CSn    & UART0 RX    & I2C0 SCL    & PWM6 B      & SIO         & PIO0        & PIO1        &              & USB VBUS DET  \\
			              &             &             &             &             &             &             &             &              &               \\ \hline
			\textbf{14}   & SPI1 SCK    & UART0 CTS   & I2C1 SDA    & PWM7 A      & SIO         & PIO0        & PIO1        &              & USB VBUS EN   \\
			              &             &             &             &             &             &             &             &              &               \\ \hline
			\textbf{15}   & SPI1 TX     & UART0 RTS   & I2C1 SCL    & PWM7 B      & SIO         & PIO0        & PIO1        &              & USB OVCUR DET \\
			              &             &             &             &             &             &             &             &              &               \\ \hline
			\textbf{16}   & SPI1 RX     & UART0 TX    & I2C0 SDA    & PWM0 A      & SIO         & PIO0        & PIO1        &              & USB VBUS DET  \\
			              &             &             &             &             &             &             &             &              &               \\ \hline
			\textbf{17}   & SPI0 CSn    & UART0 RX    & I2C0 SCL    & PWM0 B      & SIO         & PIO0        & PIO1        &              & USB VBUS EN   \\
			              &             &             &             &             &             &             &             &              &               \\ \hline
			\textbf{18}   & SPI0 SCK    & UART0 CTS   & I2C1 SDA    & PWM1 A      & SIO         & PIO0        & PIO1        &              & USB OVCUR DET \\
			              &             &             &             &             &             &             &             &              &               \\ \hline
			\textbf{19}   & SPI0 TX     & UART0 RTS   & I2C1 SCL    & PWM1 B      & SIO         & PIO0        & PIO1        &              & USB VBUS DET  \\
			              &             &             &             &             &             &             &             &              &               \\ \hline
			\textbf{20}   & SPI0 RX     & UART1 TX    & I2C0 SDA    & PWM2 A      & SIO         & PIO0        & PIO1        & CLOCK GPIN0  & USB VBUS EN   \\
			              &             &             &             &             &             &             &             &              &               \\ \hline
			\textbf{21}   & SPI0 CSn    & UART1 RX    & I2C0 SCL    & PWM2 B      & SIO         & PIO0        & PIO1        & CLOCK GPOUT0 & USB OVCUR DET \\
			              &             &             &             &             &             &             &             &              &               \\ \hline
			\textbf{22}   & SPI0 SCK    & UART1 CTS   & I2C1 SDA    & PWM3 A      & SIO         & PIO0        & PIO1        & CLOCK GPIN1  & USB VBUS DET  \\
			              &             &             &             &             &             &             &             &              &               \\ \hline
			\textbf{23}   & SPI0 TX     & UART1 RTS   & I2C1 SCL    & PWM3 B      & SIO         & PIO0        & PIO1        & CLOCK GPOUT1 & USB VBUS EN   \\
			              &             &             &             &             &             &             &             &              &               \\ \hline
			\textbf{24}   & SPI1 RX     & UART1 TX    & I2C0 SDA    & PWM4 A      & SIO         & PIO0        & PIO1        & CLOCK GPOUT2 & USB OVCUR DET \\
			              &             &             &             &             &             &             &             &              &               \\ \hline
			\textbf{25}   & SPI1 CSn    & UART1 RX    & I2C0 SCL    & PWM4 B      & SIO         & PIO0        & PIO1        & CLOCK GPOUT3 & USB VBUS DET  \\
			              &             &             &             &             &             &             &             &              &               \\ \hline
			\textbf{26}   & SPI1 SCK    & UART1 CTS   & I2C1 SDA    & PWM5 A      & SIO         & PIO0        & PIO1        &              & USB VBUS EN   \\
			              &             &             &             &             &             &             &             &              &               \\ \hline
			\textbf{27}   & SPI1 TX     & UART1 RTS   & I2C1 SCL    & PWM5 B      & SIO         & PIO0        & PIO1        &              & USB OVCUR DET \\
			              &             &             &             &             &             &             &             &              &               \\ \hline
			\textbf{28}   & SPI1 RX     & UART0 TX    & I2C0 SDA    & PWM6 A      & SIO         & PIO0        & PIO1        &              & USB VBUS DET  \\
			              &             &             &             &             &             &             &             &              &               \\ \hline
			\textbf{29}   & SPI1 CSn    & UART0 RX    & I2C0 SCL    & PWM6 B      & SIO         & PIO0        & PIO1        &              & USB VBUS EN   \\
			              &             &             &             &             &             &             &             &              &               \\ \hline
		\end{tabular}%
	}
	\captionof{table}{General Purpose Input/Output (GPIO) Bank 0 Functions}
	\small \href{https://datasheets.raspberrypi.com/rp2040/rp2040-datasheet.pdf}{Quelle: datasheets.raspberry.com}
	%\label{tabel:tab1}
\end{center}
\end{document}
